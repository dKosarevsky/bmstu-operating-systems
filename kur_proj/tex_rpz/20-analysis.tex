\chapter{Аналитический раздел}

\textbf{Разъём TRS} (аббревиатура от англ. Tip, Ring, Sleeve — кончик, кольцо, гильза; подразумевается форма контактов на штекере) — распространённый разъём для передачи аудиосигнала. \cite{wiki-trs}

Однако, наверное мало кто из тех, кто сталкивался с этими разъемами в быту слышал, что они называются TRS, а все потому что у нас прижилось другое название. Его часто называют просто <<джек>>, и это никак не связано с какой-нибудь личностью по имени Джек: на самом деле с английского языка слово <<jack>> переводится как <<гнездо>>. Причем джеком правильно называть именно гнездо, то есть куда подключается, а это разъём-мама на панели, то есть на системном блоке или другом устройстве, а разъем-папу называют plug, что и переводится как "штекер". \cite{kkg-jack}

\textbf{Джек (jack)} - это разъем диаметром 1/4 дюйма (6,3 мм). Применяется в музыкальном оборудовании, чаще всего Вы с ними встретитесь при использовании:
\begin{enumerate}
\item микрофонов для любителей (чаще всего это караоке или для домашней записи)
\item электрогитар, бас-гитар, педалей (<<примочек>>), усилителей для гитары.
\item профессиональных наушников
\item профессиональных звуковых плат
\end{enumerate}


\textbf{Мини-джек (mini-jack)} - разъём диаметром 3,5 мм. По сравнению с джеком действительно <<мини>>, и используется в устройствах, где размер действительно важен. Его вы можете встретить, купив:
\begin{enumerate}
\item наушники (обычные наушники для плеера, к примеру)
\item компьютерную акустичекую систему (её называют динамики или же колонки) 
\item гарнитуру
\item плеер
\item звуковую плату потребительского уровня
\end{enumerate}

\HRule \\
Задача делится на двек задачи.

1. Получение данных с микрофона: 
- алса
- OSS

2. Индикаторы клавы

\HRule \\
выполняется анализ поставленной задачи. Анализируются методы или способы ее решения. Проводится сравнительный анализ методов или способов решения и делается их обоснованный выбор. 
определить цель написания драйвера;

При выборе темы «Разработка драйверов для Unix и Unix-подобных систем» следует рассмотреть архитектуру подсистемы ввода/вывода, пространство имен, привести класссификацию типов драйверов и базовую архитектуру драйверов. Обоснованно выбрать тип драйвера  и определить, какой драйвер будет разрабатываться: встраиваемый или динамически загружаемый.