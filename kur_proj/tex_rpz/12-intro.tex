\Introduction

\textbf{Персональный компьютер}, так же сокращённо ПК, полагается для работы с человеком на прямую, то есть компьютер даёт возможность получить понятную информацию для человека. \cite{procomputer-pc}

Все виды персональных компьютеров (настольные компьютеры или десктопы, неттопы, моноблоки, ноутбуки, нетбуки, планшеты и планшетные ноутбуки, карманные типы компьютеров и смартфоны) взаимодействуют с пользователем посредством устройств ввода-вывода. 

\textbf{Устройство ввода-вывода} — компонент типовой архитектуры ЭВМ, предоставляющий компьютеру возможность взаимодействия с внешним миром и, в частности, с пользователями.

\textbf{Устройства ввода} — это, в основном, датчики преобразования неэлектрических величин (расположение в пространстве, давление, вязкость, скорость, ускорение, освещённость, температура, влажность, перемещение, количественные величины и т. п.) и электрических величин в электрические сигналы воспринимаемые процессором для дальнейшей их обработки в основном в цифровом виде. \cite{wiki-io}

Важнейшим устройством ввода у персональных компьютеров является микрофон.
\textbf{Микрофон} — электроакустический прибор, преобразующий акустические колебания в элект­рические коле­бания. \cite{wiki-microphone}

Все ПК оборудованы микрофонным входом \textbf{Mic In} (Microphone In).

\textbf{Устройства вывода} — это преобразователи электрической цифровой информации в вид необходимый для получения требуемого результата, могущего быть как неэлектрической природы (механические, тепловые, оптические, звуковые), так и электрической природы (трансформаторы, нагреватели, электродвигатели,реле). \cite{wiki-io}

Важнейшим устройством вывода у персональных компьютеров является акустическая система.
\textbf{Акустическая система} — устройство для воспроизведения звука, состоит из акустического оформления и вмонтированных в него излучающих головок (обычно динамических). \cite{wiki-loudspeaker_enclosure}

Все ПК оборудованы разъемом для акустичской системы (наушники, колонки).

С развитием персональных компьютеров и их минимизацией разъемы для наушников и микрофона объединились. Устройство, сочетающее микрофон и наушники, называется гарнитурой, а разъем для гарнитуры - \textbf{гарнитурный разъем}. Все мобильные устройства поддерживают гарнитуру. Обычно \textbf{в гарнитуру встраиваются кнопки}, позволяющие без прямого взаимодейсвтия с мобильным устройством быстро отвечать на звонки, изменять громкость, пролистывать музыку. 

Смартфоны и планшеты работают на \textbf{мобильных операционных системах}. Современные операционные системы для мобильных устройств: Android, CyanogenMod, Cyanogen OS, Fire OS, Flyme OS, iOS, Windows Phone, BlackBerry OS, Firefox OS, Sailfish OS, Tizen, Ubuntu Touch. Устаревшие, ныне не поддерживаемые программные платформы: Symbian, Windows Mobile, Palm OS, webOS, Maemo, MeeGo, LiMo. \cite{wiki-mob_os}

Многие мобильные операционные системы поддерживают работу с кнопками гарнитуры. Эти ОС предоставляют удобный API для обработки нажатий кнопок. Существует большое колличество программ для мобильных устройств, использующие API от операционной системы. Например: \textbf{Headset Button Controller} - управление музыкальным проигрывателем и другими функциями телефона с проводной гарнитуры, превращает проводную гарнитуру в пульт дистанционного управления для вашего телефона. Работает как с 1-но кнопочной, так и с 3-х кнопочными гарнитурами. Действия можно программировать на любую из клавиш. Аналогичные программы: Headset Droid, Headset Volume Controller, JAYS Headset Control и Philips Headset. Пример кода на ОС Android представлен в листинге \ref{ExampleAndroid} \cite{stackoverflow-android_push}.

\begin{lstlisting}[language=Java, caption={Пример обработки кнопки гарнитуры},label=ExampleAndroid]
boolean onKeyDown(int keyCode, KeyEvent event) { 
    AudibleReadyPlayer abc; 
    switch (keyCode) { 
    case KeyEvent.KEYCODE_MEDIA_FAST_FORWARD: 
            // code for fast forward 
            return true; 
    case KeyEvent.KEYCODE_MEDIA_NEXT: 
            // code for next 
            return true; 
    case KeyEvent.KEYCODE_MEDIA_PLAY_PAUSE: 
            // code for play/pause 
            return true; 
    case KeyEvent.KEYCODE_MEDIA_PREVIOUS: 
            // code for previous 
            return true; 
    case KeyEvent.KEYCODE_MEDIA_REWIND: 
            // code for rewind 
            return true; 
    case KeyEvent.KEYCODE_MEDIA_STOP: 
            // code for stop 
            return true; 
    } 
    return false; 
} 
\end{lstlisting}

Также существуют альтернативные способы использования гарнитурного разъема. Периферийные устройства, например, глюкометр iHealth Lab (определяющий уровень сахара в крови), Irdroid – ИК-пульт для дистанционного управления телевизором, приставками и звуковыми компонентами и Flojack – устройство чтения NFC (организующее радиосвязь между находящимися рядом мобильными устройствами) расширяют возможности мобильных устройств, используя гарнитурный разъем.

Однако операционные системы Linux и Windows, ориентрированные на десктопы и ноутбуки, не поддерживают работу с гарнитурой. 

В статье \cite{gambiarraemcasa} автор собирает осцилограф и пользуется программой xoscope. Xoscope для Linux - цифровой осцилограф, который использует звуковую карту. Это единственное нестандартное использование микрофоного входа компьютера на не мобильной операционной системе.

В данной работе ставится задача написания программы, которая бы обнаруживала в ОС Linux нажатие кнопоки на гарнитуре.  

Другим важнейшим устройством вывода у персональных компьютеров является \textbf{монитор или дисплей}.

Если в системе возникает критическая ошибка, она можеть быть выведена на экран или записана в логи. Иногда компьютер работает без дисплея, а возможности прочитать лог-файлы нет, так как диск зашифрован, а ошибка случается до расшировки. Из данной ситуации можно выйти, используя альтернативное взаимодействие с пользователем.

Когда в Linux случается kernel panic, ошибка может быть выведена через LED индикатор клаиатуры, используя азбуку Морзе. \cite{lwn-mckp}

\textbf{BIOS} (Basic Input/Output System – базовая система ввода-вывода) - пограмма системного уровня, предназначенная для первоначального запуска компьютера, настройки оборудования и обеспечения функций ввода/вывода. ВIOS записывается в микросхему постоянной памяти, которая расположена на системной плате. В персональных и портативных компьютерах система BIOS производит запуск компьютера и процедуру самотестирования (Power-On Self Test – POST). Программа, расположенная в микросхеме BIOS, загружается первой после включения компьютера. Программа определяет и проверяет установленное оборудование, настраивает устройства и готовит их к работе. При самотестировании, возможно, будет обнаружена неисправность оборудования, тогда процедура POST будет остановлена с выводом соответствующего сообщения или звукового сигнала. Если же все проверки прошли успешно, самотестирование завершается вызовом встроенной подпрограммы для загрузки операционной системы. Ну а если же программой будет выявлена серьезная ошибка, работа системы будет остановлена с выдачей звуковых сигналов, которые укажут на возникшую неисправность. \cite{allmbs-bios}

Т.о. на низком уровне большую роль играет альтернативное взаимодействие компьютера с пользователем. Поэтому в данной работе разберем информирование пользователя компьютера LED индикаторами на уровне ядра в ОС Linux.

