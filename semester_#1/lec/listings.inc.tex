% 8 Листинги

\usepackage{listings}

% Значения по умолчанию
\lstset{
  basicstyle= \footnotesize,
  breakatwhitespace=true,% разрыв строк только на whitespacce
  breaklines=true,       % переносить длинные строки
%   captionpos=b,          % подписи снизу -- вроде не надо
  inputencoding=utf8x,
  extendedchars=false,
  keepspaces=true,
  numbers=left,          % нумерация слева
  numberstyle=\footnotesize,
  showspaces=false,      % показывать пробелы подчеркиваниями -- идиотизм 70-х годов
  showstringspaces=false,
  showtabs=false,        % и табы тоже
  stepnumber=1,
  tabsize=4,              % кому нужны табы по 8 символов?
  frame=single
}

% Стиль для псевдокода: строчки обычно короткие, поэтому размер шрифта побольше
\lstdefinestyle{pseudocode}{
  basicstyle=\small,
  keywordstyle=\color{black}\bfseries\underbar,
  language=Pseudocode,
  numberstyle=\footnotesize,
  commentstyle=\footnotesize\it
}

% Стиль для обычного кода: маленький шрифт
\lstdefinestyle{realcode}{
  basicstyle=\scriptsize,
  numberstyle=\footnotesize
}

% Стиль для коротких кусков обычного кода: средний шрифт
\lstdefinestyle{simplecode}{
  basicstyle=\footnotesize,
  numberstyle=\footnotesize
}

% Стиль для BNF
\lstdefinestyle{grammar}{
  basicstyle=\footnotesize,
  numberstyle=\footnotesize,
  stringstyle=\bfseries\ttfamily,
  language=BNF
}

% Определим свой язык для написания псевдокодов на основе Python
\lstdefinelanguage[]{Pseudocode}[]{Python}{
  morekeywords={each,empty,wait,do},% ключевые слова добавлять сюда
  morecomment=[s]{\{}{\}},% комменты {а-ля Pascal} смотрятся нагляднее
  literate=% а сюда добавлять операторы, которые хотите отображать как мат. символы
    {->}{\ensuremath{$\rightarrow$}~}2%
    {<-}{\ensuremath{$\leftarrow$}~}2%
    {:=}{\ensuremath{$\leftarrow$}~}2%
    {<--}{\ensuremath{$\Longleftarrow$}~}2%
}[keywords,comments]

% Свой язык для задания грамматик в BNF
\lstdefinelanguage[]{BNF}[]{}{
  morekeywords={},
  morecomment=[s]{@}{@},
  morestring=[b]",%
  literate=%
    {->}{\ensuremath{$\rightarrow$}~}2%
    {*}{\ensuremath{$^*$}~}2%
    {+}{\ensuremath{$^+$}~}2%
    {|}{\ensuremath{$|$}~}2%
}[keywords,comments,strings]

\lstdefinelanguage[]{sqlite}[]{sql}{
  deletekeywords={POSITION},
  morekeywords={total},
  commentstyle=\color{gray}
}

\lstdefinelanguage[]{sql-explain}[]{sqlite}{
  morekeywords={SCAN, SEARCH},
  breaklines=false
}

% Подписи к листингам на русском языке.
\renewcommand\lstlistingname{\cyr\CYRL\cyri\cyrs\cyrt\cyri\cyrn\cyrg}
\renewcommand\lstlistlistingname{\cyr\CYRL\cyri\cyrs\cyrt\cyri\cyrn\cyrg\cyri}
