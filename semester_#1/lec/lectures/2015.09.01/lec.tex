\clearpage
\begin{flushright}
	\textit{Лекция №1}
	\textit{2015.09.01}
\end{flushright}

План занятий:
\begin{enumerate}
    \item 1 семестр:
    \begin{enumerate}
        \item управление процессорами;
        \item управление памятью;
        \item взаимодейсвтие процессов.
    \end{enumerate}
    \item 2 семестр:
    \begin{enumerate}
        \item управление данными(файловые системы, доступ к данным);
        \item управление устройствами;
    \end{enumerate}
\end{enumerate}

\chapter{История вычислительной техники}

Выч. машина – программно-управляемое устройство. 
Без человека – автоматическое управление. С человеком – арифмометр ???.
Реле подтолкнуло к двоичным вычислениям.
Марк 1 – электромагнитное реле 1944 года Тьюринга. Фон Нейман – первое серийное.

\section{Первое поколение 1944 – 1955} 

Сформулированы принципы архитектуры Фон Неймана:
\begin{enumerate}
    \item Основные блоки: блок управления, АЛУ, запоминающее устройство и устройство ввода – вывода;
    \item Программы и данные хранятся в одной и той же памяти, таким образом концепция хранимой программы является основной;
    \item Устройство управления и АЛУ обычно объединяются в центральный процессор и определяют действия, подлежащие выполнению путем считывания команд из оперативной памяти. Следует что программа состоит из команд, которые проверяются одна за другой;
    \item Адрес очередной ячейки памяти, из которой следует брать команду указывается счетчиком команд в устройстве управления. Следует, что данные, с которыми работает программа, могут включать в себя переменные: области памяти могут быть поименованны, так что в заполненным в них значениях можно обращаться к ним и менять по присвоенным им именам.
\end{enumerate}

Следствия:
\begin{enumerate}
    \item ОЗУ по быстродействию должно быть сопоставимо с быстродействием процессора.
    \item Программа в памяти располагается команда за командой. Команды программы располагаются в последовательных адресах. Для обращения к очередной команды мы используем счетчик команд (убрали адрес следующей команды, появление условных переходов).
    \item В машине сигналы трех типов: данные (команды и собственно данные), адреса, сигналы управления.
\end{enumerate}

\section{Второе поколение (транзисторная логика).}

Меняются элементы памяти (магнитные сердечники). Магнитные ленты – средство хранения данных. Магнитные барабаны
Серия – полная документация на машину. Что бы что-то запустить в серию должны быть созданы чертежи железа.
Ассемблер – мнемоническое оформление машинных кодов. 

\section{Третье поколение. 60-е}

Появление Интегральных микросхем. 
1 шаг – загрузить программу в оперативную память. 
Чтобы уменьшить время простоя процессора нужно загрузить в память несколько программ (мультипрограммный режим пакетной обработки) программы в памяти находятся целиком и выполняются от начала до конца или до ошибки. Операционные системы заменяли человека оператора, который должен был загружать в память программы с соответствующими программами и библиотеками. Переключение с одной программы на другую осуществляется автоматически операционной системой. Но процессор простаивает в ожидании ввода вывода данных.

Операционная система – комплекс программ для управления процессами и выделения процессам ресурсов вычислительной машины.
Процесс – программа в стадии выполнения. Программа на диске – файл.

Программы бывают: 
\begin{enumerate}
    \item исходник (написан на языке высокого уровня);
    \item объектный (получается трансляторами, трансляторы многопроходные и оптимизирующие), 
    \item экзешники (исполняемые программы получаются ликновщиками, они подключают файлы библиотек). Исполняемая программа может выполняться только средствами операционной системы.
\end{enumerate}

К процессору организуется очередь процессов. Прежде чем процесс сможет начать выполняться ему должно быть выделен ресурс – память. 

\subsection{Канальная архитектура}

Семейство  IBM/360:
Возникла идея распараллеливания функций. Появляется архитектура выч. машин. В состав ибм360 включаются ??? название канала. Такая архитектура называется канальной. Канал – программно-управляемое устройство, получающее от процессора канальную программу, под управлением которой канал берет на себя управление внешним устройством. Т.е. внешним устройством управляет не цп, а канал. Процессор инициирует управление внешним устройством. Непосредственное управление осуществляется каналом. Процессор может переключиться на другую программу, пока не получит управляющий сигнал, что операция ввода-вывода завершена. Появилась система прерываний.


